\documentclass[10pt]{article}
\title{CS346 - Asg 4 - David Ko - dpk326}






\begin{document}
\date{}
\maketitle

\noindent \large \textbf{Problem 1}\\
\indent I am a super senior after spending the first 3 years of my college career miserable as an electrical engineer and now I am finally going to graduate this May. Originally I took AP comp sci in high school and learned basic html and thought it was pretty easy so after I left EE,  thought I would give CS a try but it was not the same, but it was at least enjoyable. No, I am not a Turing Scholar sadly and I am in social organizations like VSA.\\

\noindent \large \textbf{Problem 2}\\
a. This is a group because it is closed under closure; has the associative property; it has an identity element which would be the zero matrix since a matrix + a matrix full of zeros would equal itself; and it has an inverse.\\

\noindent b. This is not a group. It fulfills closure, associativity, and identity element, but it does not the inverse the 2x2 matrix wit only the real number '2' does not have an inverse since the determinant is 1/0.\\

\noindent c. This is not a group. For a matrix to have an inverse, all entries in the matrix must be \emph{all} non-zero.\\


\noindent \large \textbf{Problem 3}\\
a. GCD(2015,5797)\\
\indent -want to find the GCD of 2015 and 5797\\
\indent -swap the numbers to make $a>b$ so it becomes GCD(5797,2015)\\
\indent 5797 mod 2015 = 1767\\
\indent 2015 mod 1767 = 248\\
\indent 1767 mod 248 = 31\\
\indent So the GCD is 31.\\

\noindent b. $15^{-1}$ (mod 346)\\
\indent Here we set it up like the Euclidean algorithm, but for inverse:\\
\indent $346 = 15 * (x) + 1$\\
\indent Calcuating for x,\\
\indent $346 = 15 * (23) + 1$\\
\indent Now, working backwards we get:\\
\indent $346 - 345 = 1$\\
\indent Now subbing for 346:\\
\indent $(15*23 + 1) - 345 = 1$\\
\indent Now subbing for 345 we get:\\
\indent $(15*23 + 1) + 15*(-23) = 1$ (mod 346)\\
\indent So we take 346 - 23 = 323 and we sub it back in for -23:\\
\indent $(15*23 + 1) + 15*(323) = 1$ (mod 346)\\
\indent As we can see, the first part $(15*23 + 1)$ mod 346 goes to 0, so we can remove it,\\
\indent So we are left with $15*(323) = 1$ (mod 346),\\
\indent And now we know that the answer to $15^{-1}$ (mod 346) is 323.\\



\noindent \large \textbf{Problem 4}\\
a. Subgroups: The group itself and \{1\} are the trivial groups so they are subgroups, \{1,2,3\}, \{1,2,4\}, \{1,2,5\}, \{1,2\}, \{1,3\}, \{1,4\}, \{1,5\}, 
\{1,6\}, \{1,7\}, \{1,8\}, \{1,9\}, \{1,10\}.\\


\noindent b. Order of the group is $\Phi(p) = p - 1 = 11 - 1 = 10$.\\

\noindent c. The generators of this group are 2, 6 and 8.\\
\indent The way I calculated this was I was through each number in the group and figured out if it hit all members:
\indent For example for 8: $\textbf{8}\rightarrow64$ mod 11 = \textbf{9}$\rightarrow$72 mod 11 = \textbf{6}$\rightarrow$48 mod 11 = \textbf{4}$\rightarrow$32 mod 11 = \textbf{10}$\rightarrow$80 mod 11 = \textbf{3}$\rightarrow$24 mod 11 = \textbf{2}$\rightarrow$16 mod 11 = \textbf{5}$\rightarrow$40 mod 11 = \textbf{7}$\rightarrow$56 mod 11 = \textbf{1}\\

\noindent \large \textbf{Problem 5}\\
\noindent a. $7^{2015346001}$ mod 11.\\
\indent Using the formula on the powerpoint lectures we get:\\
\indent $7^{(201534600 * 10) + 1}$ with $\alpha$ being $0 \leq 1 \alpha = 1 \leq 10$\\
\indent $7^{(201534600 * 10)} * 7^{1}$\\
\indent = $7^{1}$\\

\noindent b. $15^{23} (mod 1051)$ and use repeated squaring.\\
\indent We first find the powers of 2 that make up 23 which are:\\
\indent 16, 4, 2, and 1.\\
\indent So we have $15^{16+4+2+1} (modulo 1051)$\\
\indent Split up: $15^{16} (modulo 1051) * 15^{4} (modulo 1051) * 15^{2} (modulo 1051) * 15^{1} (modulo 1051)$\\
\indent $15^{1} (modulo 1051) = 15$. Then we take $15^{2} (modulo 1051) = 225$.\\
\indent Now we take $225^{2} (modulo 1051)$ to get $15^{4} (modulo 1051)$ which is 177.\\
\indent To get $15^{8} (modulo 1051)$ we take $15^{4} (modulo 1051)$ and then square it then take the (modulo 1051) which is $177^{2} (modulo 1051) = 850$.\\
\indent To get $15^{16} (modulo 1051)$, we take $850^{2} (modulo 1051) = 463$.\\
\indent Now we take the values 15, 225, 177, and 463 for the respective square 1, 2, 4, 16 and we multiply them together and take the modulo 1051:\\
\indent $(15 * 225 * 177 * 463) (modulo 1051) = 312$.\\
\indent So $15^{23} (modulo 1051) = 312$.\\ 

\noindent \large \textbf{Problem 6}\\
\indent The elements with $p = 35$ are \{1, 2, 3, 4, 5, 6, 7, 8, 9, 10 , 11, 12,13, 14, 15, 16, 17, 18, 19, 20, 21, 22, 23, 24, 25, 26, 27, 28, 29, 30, 31, 32, 33, 34\}. Basically 1 to p-1.\\


\noindent \large \textbf{Problem 7}\\
\indent The order of a group $\emph{G}$ is defined as the size of $|G|$ of the set $\emph{G}$\\
\indent Assume there exists an element $\emph{a} \in \emph{G}$ with order greater than \emph{k}. Consider the set \emph{\{a, $a^{2}$,...$a^{k}$\}} - by our assumption, since the order of \emph{a} is greater than \emph{k}, none of these elements are unity, therefore there exist \emph{i,j} $\leq$ \emph{k,i} $\neq$ \emph{j} such that \emph{$a^{i}$ = $a^{j}$}.\\
\indent If there exists an element $\emph{a} > \emph{k}$, for example lets say the order of a group \emph{G} is $\emph{k = 15}$ and that there exists an element \emph{a} that is 18, which is $>$ \emph{k}. This is a straigh contradiction of the definition of order of a group which is the cardinality of the group \emph{G}, meaning the elements of the group \emph{G} are from 0 to 15 and that the element \emph{a} which is 18 is not included. Therefore, there cannot be an element \emph{a} that is $>$ \emph{k}.\\ 


\noindent \large \textbf{Problem 8}\\
\indent I really enjoy private key encryption because its simplistic and only requires one key to be shared between the users and not different keys making it more straight forward than public key encryption.\\




\end{document}