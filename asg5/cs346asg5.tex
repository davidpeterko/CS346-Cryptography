\documentclass[10pt]{article}
\title{CS346 - Asg 5 - David Ko - dpk326}

\begin{document}
\maketitle
\date{}


\noindent \large \textbf{Problem 1}\\
(a) Given \textbf{ct}, we must create a different ciphertext \textbf{ct'} that, when decrypted, must be the same message \emph{M}.\\
\indent To get \textbf{ct}, we let \emph{$c_{1}$ := $g^{y}$, $c_{2}$ := $h^{y}m$}. The \emph{pk} is generated by \emph{(p,g,h)} and \emph{q} by \emph{(p-1)/2}. And we know that Dec(\emph{sk}, \emph{ct}) = \emph{M}.\\

\indent So, lets say we uniformly choose at random another \emph{x} $\in$ \textbf{$Z_{q}$} and generate a new \emph{g} to get a new \emph{(p,g)} generated by the Gen($1^{n}$). With the new \emph{x}, we get \emph{h} := \emph{$g^{x}$} and a new \emph{pk} := \emph{(p,g,h)} and \emph{sk} := \emph{(p,g,x)}.\\
\indent Now we uniformly choose at random \emph{y} $\in$ \textbf{$Z_{q}$} to generate a new \emph{$c_{1}$} := \emph{$g^{y}$} and \emph{$c_{2}$} := \emph{$h^{y}m$} to return a new ciphertext \textbf{ct'}.\\
\indent Now we have constructed a new and \emph{different} ciphertext \textbf{ct'} such that $\textbf{ct} \neq \textbf{ct'}$.\\
\indent We have re-randomized the ciphertext using new exponents from \textbf{$Z_{q}$} and a new generator from \emph{G}.\\
\indent By re-randomizing the new ciphertext, we can efficiently and continuously construct a new ciphertext such that Dec(\emph{sk}, \emph{ct'}) = Dec(\emph{sk}, \emph{ct}) = \emph{M}.\\

(b) We must construct an adversary that shows that El Gamal is not IND CCA secure.\\
\indent 1. We have an attacker and the challenger.\\
\indent 2. The challenger Gen($1^{n}$) the public key and gives it to attacker as well as the decryption oracle. The challenger also chooses \emph{x} $\in$ \textbf{$Z_{q}$} uniformly at random to generate \emph{h} := \emph{$g^{x}$}.\\
\indent 3. The attacker then queries 2 messages $m_{0}$ and $m_{1}$ and sends them to the challenger.\\
\indent 4. The challenger then choses a bit \emph{b} $\in$ {0,1} uniformly at random and a \emph{y} $\in$ \textbf{$Z_{q}$} and generates the ciphertext \emph{ct} of $m_{b}$. The ciphertext is ($c_{1}$, $c_{2}$) where $c_{1}$ := $g^{y}$, $c_{2}$ := \emph{$h^{y}$m} ($c_{2}$ can also be rewritten as \emph{$g^{xy}$m} since \emph{h} := \emph{$g^{x}$}) and then sends it to the attacker. With the ciphertext, the attacker can choose a random number \emph{z} and modify the ciphertext to get \emph{ct'} where $c_{1}$ := $g^{y+z}$, $c_{2}$ := \emph{$g^{(y+z)x}$m} since the attacker now knwos the values of \emph{g} and \emph{$g^{y}$}.\\
\indent 5. The attacker then can use the decryption oracle to decrypt this ciphertext \emph{ct'} to get the original message \emph{$m_{b}$}.\\

\indent The attackers advantage in the IND-CCA game is 1 since the can get the value of \emph{a} for the \emph{sk} to decrypt to $m_{b}$.\\

\noindent \large \textbf{Problem 2}\\
\indent 1. There is an attacker and a challenger.\\
\indent 2. The challenger runs Gen($1^{n}$) to generate \emph{(p,g)} and \emph{q} from (p-1)/2. The challenger chooses \emph{x} $\in$ \textbf{$Z_{q}$} uniformly at random to get \emph{h} := \emph{$g^{x}$}. The challenger then sends the public key and decryption oracle to the attacker.\\
\indent 3. The attacker than sends 2 messages, $m_{0}$ and $m_{1}$ to the challenger.\\
\indent 4. The challenger then Enc(\emph{pk,M}) by letting \emph{M} = ($M_{x}$, $M_{y}$) $\in$ \textbf{G} X \textbf{G}. The challenger then chooses a random \emph{r} $\in$ \textbf{$Z_{q}$}. The ciphertext generated is $C_{1}$ = $g^{r}$ (like the unmodified El Gamal), $C_{2}$ = $A^{r}$$M_{x}$, and $C_{3}$ = $A^{r}$$M_{y}$. The message \emph{M} is split into two messages $M_{x}$ and $M_{y}$ in this modified El Gamal.\\
\indent 5. This part is where the El Gamal fails the DDH assumption. $C_{2}$ and $C_{3}$ can be combined by the attacker to generate a $C_{x}$ where $C_{x}$ = 2($A^{r}$)M. This is not IND CPA secure because the security of El Gamal depends on the security of G. G must be a prime order group and when you take the cross product \textbf{G} X \textbf{G} this fails, this breaks the security of the El Gamal encryption.\\
\indent 6. The attacker then recieves the ciphertext, and returns b which would indicate $m_{b}$.\\

\indent The DDH assumption does not hold, therefore this modified EL Gamal is not secure.\\

\indent The advantage of the attacker is Adv(A) = Pr[b=b'|d = 0] = $\frac{1}{2}$ + $\varepsilon$.\\
\\


\noindent \large \textbf{Problem 3}\\
\indent In this problem, if the scheme is two time secure, then it is also one time secure. But if we construct a PPT \emph{A} that breaks the two time secure scheme, then it also breaks the one time secure scheme.\\

\indent 1. The challenger does Setup($1^{n}$) and sets the verification key to \textbf{$vk_{1}$} and the signing key as \textbf{$sk_{1}$}. The challenger then sends the  Signing oracle to the challenger.\\
\indent 2. The attacker than sends a message \emph{$m_{0}$} $\in$ ${1,0}^{n}$ to the challenger.\\
\indent 3. The challenger signs the message using Sign(\textbf{$sk_{1}$}, \emph{m}). Since this is the first message to be signed, the challenger then updates the signing key and verification key and chooses ($sk_{2}$, $vk_{2}$) and computes $\sigma$ signature. After it signs the message \emph{$m_{0}$}, it updates the \textbf{st} to {(m, $vk_{2}$, $sk_{2}$, $\sigma_{1}$)} and then sends the signature (\emph{m}, $vk_{2}$, $\sigma$) to the attacker.\\ 
\indent 4. The attacker then creates a forgery and then sends it to the challenger to run Verify($vk_{1}$, \emph{T}, \emph{m}). The challenger then outputs 0 if \emph{m} $\neq$ \emph{$m_{i}$}. Else, 1.\\

\indent This is where the one-time secure scheme ends since the attacker at max can make 1 query. This scheme is one-time secure, but lets look at if the attacker can make another query.\\

\indent The attacker sends a second message $m_{1}$ to the challenger.\\

\indent The challenger recognizes that this is not the first message sent, and updates the verification key and signing key to \textbf{$vk_{3}$} and the \textbf{$sk_{3}$}.\\

\indent What happens if $m_{1}$ is the same as $m_{0}$? Unlike Encrypt then MAC where if the same message is encrypted twice, the MAC part will be identical in both cases, since the state is updated the signing key and verficition key are not different from the prevous state, there fore avoiding the problem Encrypt Then MAC has.\\

\indent The attacker now has to deal with a new signing key and verification key \textbf{$vk_{3}$} and the \textbf{$sk_{3}$} during this second query, so the result will be different from the previous query sincen the state is updated and keys are also updated.\\
\indent From this we can see that this scheme is two-time secure.\\

\indent Because the assumption that if this scheme is two-time secure, then it must also be one-time secure. If we can construct a PPT attacker that breaks the two-time secure scheme, then that PPT attacker can also break the one-time secure secure since a two-time secure scheme $\to$ the scheme is also one-time secure.\\

\noindent \large \textbf{Problem 4}\\
(a) Construct the encryption scheme $\varepsilon$.\\

\indent \textbf{Gen($1^{n}$)} runs Setup and generates the public key and the secret key \emph{(pk, sk)}.\\
\indent \textbf{Enc(pk, m)} takes the public key \emph{pk} and a message \emph{m} $\in$ \emph{M} as input and outputs a ciphertext \emph{ct} $\in$ \emph{C} where \emph{ct} = (pk,m).\\
\indent 
\indent \textbf{Dec(sk, ct)} takes in the secret key \emph{sk} and te cipher text and outputs a message \emph{m} $\in$ \emph{C}.\\

\indent Given that $\varepsilon'$ is IND-CPA secure, we take a look at $\varepsilon$ and see how secure it is.\\

\indent We prove that $\varepsilon'$ is IND-CPA secure by playng the security game.\\
\indent 1. The challenger generates a key k by running KeyGen(n) during Setup($1^{n}$), where n is the security parameter.\\
\indent 2. We give the attacker Oracle access to the Enc(k).\\
\indent 3. The attack sends back 2 messages $m_{0}$ and $m_{1}$.\\
\indent 4. The challenger than choose a bit \emph{b} $\in$ {0,1} uniformly at random and send the cipher text \emph{c} from running Enc(k,$m_{b}$) back to the attacker.\\
\indent 5. The attacker is given Oracle Access to Enc(k,m) and the attacker outputs a guess \emph{b'}.\\
\indent The scheme $\varepsilon'$ is secure if the Adv of the attacker is neglible. \emph{negl(n)}.\\

\indent So we see that $\varepsilon'$ is IND-CPA secure from the security game under those conditions having a \emph{negl(n)} advantage. Since it is IND-CPA secure, it is also IND-Random secure. So if $\varepsilon'$ is IND-CPA secure, then we know that $\varepsilon$ is IND-Random secure.\\
\indent By contradiction, if there exists a PPT A that breaks $\varepsilon'$, then there exists a PPT B that breaks $\varepsilon$ and $\varepsilon$ is no longer IND-Random secure.\\

\indent Showing that $\varepsilon$  is not IND-CPA secure:\\

\indent The attacker will win with an advantage of 1/2. Why is this?\\

\indent 1. The challenger runs Setup and generates a public key and secret key, \textbf{pk} and \textbf{sk} respectively. The challenger then sends the \textbf{pk} to the attacker.\\
\indent 2. The attacker then generates 2 message $m_{0}$ and $m_{1}$ uniformly at random from the message space and sends it to the challenger. The Challenger then choses a bit \emph{b} at random from ${0,1}$.\\
\indent 3. The challenger then runs Enc and generates a ciphertext \emph{ct} from Enc(pk,m) and then sends it to the attacker.\\
\indent 4. The attacker did not himself generate the sk so he has no knowledge of what $m_{0}$ or $m_{1}$ is from just the ciphertext. The attacker just knows what the \textbf{pk} is and what the \emph{ct} is. Because of this the attacker must just choose b' from either 0 or 1 because the attacker has no knowledge of what message was being used.\\
\indent 5. The attacker sends back \emph{b'}. The challenger ouputs 0 if b $\neq$ b' or 1 if b = b'.\\

\indent This is not IND-CPA secure because the attacker can beat the challenger 50\% of the time. The attacker just has to choose either 0 or 1 and the attacker will get \emph{b} right half the time resulting in an adv of:\\

\indent Adv[A] = (Pr[b = 0 | b' = 0] + Pr[b = 1 | b' = 1]) -  (Pr[b = 0 | b' = 1] + Pr[b = 1 | b' = 0]) = $\frac{1}{2}$.\\

(b) This scheme differs from the previous one in part (a) because in part (a) the message space was much smaller, allowing the message only to be either one or the other resulting in a advantage of 1/2.\\

\indent But in this scheme, this message space is {0, 1, 2, ... n-1} which means that if the previous scheme has an adv of $\epsilon$, then for each addition message available in the message space, the adv of the attacker would be reduce to $\frac{\epsilon}{n^{2}}$.\\

\indent WHy is this? Lets take a look:\\
\indent In the IND-CPA scheme before, the attacker would generate 2 messages $m_{0}$ and $m_{1}$ and send it to the challenger to encrypt. But since the message space is much larger in this currentn scheme, we would have to send the messages $m_{0}$, $m_{1}$, $m_{2}$... $m_{n-1}$ thus reducing the adv of the attacker.\\
\indent In the random game, the challenger generates 2 messages itself, so the attacker has no idea what they are; the attackerk only gets back the ciphertext, so all the attacker can do it guess either 0 or 1, but in this scheme, the attacker gets back too many messages and the adv of the attacker becomes:\\

\indent Adv[A] = SUMMATION OF{(Pr[b = 0 | b' = 0] + Pr[b = 1 | b' = 1]) + Pr[b = 3 | b' = 3])... Pr[b = n-1 | b' = n-1])} -  SUMMATION OF (Pr[b = 0 | b' = 1] + Pr[b = 0 | b' = 2] + Pr[b = 0 | b' = 3].. and so on)) = which converges to $\frac{\epsilon}{n^{2}}$.\\


\end{document}